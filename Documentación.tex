\documentclass{article}
\usepackage{graphicx}
\usepackage[margin=3cm]{geometry}
\usepackage{subcaption}
\usepackage[spanish,es-tabla]{babel}
\usepackage[utf8]{inputenc}
\usepackage{url}
\usepackage{tcolorbox}
\usepackage{listings}




\title{Documentación Taco Paco}
\author{Santi Martínez}
\date{\today}

\begin{document}
    \maketitle
    lstset{languaje=Javascript}
    \newpage
    \tableofcontents

    \newpage
    \section{Introducción}
        Taco Paco es una aplicación de comida rápida en la que el cliente puede reservar su mesa, y a partir de ahí seleccionar sus productos favoritos.
        \\\\
        La app de escritorio forma parte únicamente del establecimiento, ya que su funcionalidad radica en cobrar pedidos y liberar las mesas. Gracias a ello podrá ver también las mesas libres y las ocupadas.
        \\\\
        Ambas se conectan entre sí a través de una API Rest elaborada con express y mongoose, el último para guardar y editar los cambios en Mongo Atlas, la base de datos de la aplicación.

    \newpage
    \section{Arquitectura de la app}
        \subsection*{\underline{Diagrama de flujo}}
            \begin{figure}[h!]
                \centering
                \includegraphics[width=0.9\textwidth]{images/Diagrama-flujo.png}
                \caption{Diagrama de flujo}
            \end{figure}
        \newpage
        \subsection*{\underline{Diagrama de grafo}}
            \begin{figure}[h!]
                \centering
                \includegraphics[width=0.9\textwidth]{images/Diagrama-grafo.png}
                \caption{Diagrama de grafo}
            \end{figure}

        \newpage
        \subsection*{\underline{Diagrama de casos de uso}}
            \begin{figure}[h!]
                \centering
                \includegraphics[width=1.0\textwidth]{images/Diagrama-casos-de-uso.png}
                \caption{Diagrama de casos de uso}
            \end{figure}

    \newpage
    \section{APP Móvil}
        \subsection*{Estructura de la app}
        \begin{itemize}
            \item app/
            \begin{itemize}
                \item sampledata/
                \item manifest/
                \begin{itemize}
                    \item \texttt{AndroidManifest}
                \end{itemize}
                \item kotlin+java/
                \begin{itemize}
                    \item \textbf{activities/}
                    \begin{itemize}
                        \item \texttt{Carta}
                        \item \texttt{EleccionMesa}
                        \item \texttt{MainActivity}
                    \end{itemize}
                    \item \textbf{clases/}
                    \begin{itemize}
                        \item \texttt{Mesa}
                        \item \texttt{Pedido}
                    \end{itemize}
                    \item \textbf{servicios/}
                    \begin{itemize}
                        \item \texttt{Api}
                        \item \texttt{RetrofitClient}
                    \end{itemize}
                \end{itemize}
                \item java (generated)/
                \item resources/
                \begin{itemize}
                    \item \textbf{drawable/}
                    \item \textbf{layout/}
                    \begin{itemize}
                        \item \texttt{carta}
                        \item \texttt{eleccion mesa}
                        \item \texttt{inicio}
                    \end{itemize}
                    \item \textbf{mipmap/}
                    \item \textbf{values/}
                    \item \textbf{xml/}
                \end{itemize}
                \item resources (generated)/
            \end{itemize}
            \item Gradle Scripts/
        \end{itemize}

        \newpage
        \subsection*{Imágenes de la app}

            % == Pantalla de inicio ==
            \begin{minipage}{.5\textwidth}
                \vspace{0.5cm}
                \textbf{\begin{center}Pantalla principal\end{center}}
                \vspace{0.5cm}

                Primera pantalla que se encuentra el usuario al iniciarl la app, la única funcionalidad que tiene es la que le da el botón \textbf{Entrar}, el cual como su nombre indica, sirve para entrar en el siguiente menú de la app.
                \vspace{0.5cm}

                En la pantalla se encuentra el nombre de la empresa junto a su año en la que se fundó. También el botón de entrar que es el encargado de cambiar entre activities.
            \end{minipage}
            \hspace{1cm}
            \begin{minipage}{.4\textwidth}
                \includegraphics[scale=0.3]{images/pantalla_principal.png}
            \end{minipage}

            % == Pantalla de mesas (TODAS LIBRES) ==
            \begin{minipage}{.4\textwidth}
                \includegraphics[scale=0.3]{images/pantalla_mesas.png}
            \end{minipage}
            \begin{minipage}{.5\textwidth}
                \vspace{0.5cm}
                \textbf{\begin{center}CASO 1: Todas las mesas libres\end{center}}
                \vspace{0.5cm}

                Esta es la pantalla en la que se muestran las mesas disponibles del establecimiento. Se trata de un menú con 5 mesas:
                \begin{itemize}
                    \item 3 interiores
                    \item 2 terraza
                \end{itemize}
                Esta diferenciación se encuentra en la parte baja de la pantalla, la cual se destaca con dos colores diferentes.

                En pantalla se encuentran los botones que llevan a los activities de sus respectivas mesas y un botón para volver al activity anterior.
            \end{minipage}

            % == Pantalla de mesas (CON OCUPADAS) ==
            \begin{minipage}{.5\textwidth}
                \vspace{0.5cm}
                \textbf{\begin{center}CASO 2: Mesas ocupadas\end{center}}
                \vspace{0.5cm}

                Mismo funcionamiento que en el caso anterior, pero sin embargo en esta pantalla se pueden apreciar que tanto la \textbf{Mesa 3} como la \textbf{Mesa 4}, se encuentran ocupadas.
                
                La app no va a permitir su ocupación ni utilización de las mismas, es decir, quedan bloqueadas hasta que desde la APP-Escritorio se libreren cobrando el pedido a los clientes.
            \end{minipage}
            \hspace{1cm}
            \begin{minipage}{.4\textwidth}
                \includegraphics[scale=0.3]{images/pantalla_mesas_ocupadas.png}
            \end{minipage}

            % == CARTA ==
            \begin{minipage}{.4\textwidth}
                \includegraphics[scale=0.3]{images/pantalla_carta.png}
            \end{minipage}
            \begin{minipage}{.5\textwidth}
                \vspace{0.5cm}
                \textbf{\begin{center}Carta\end{center}}
                \vspace{0.5cm}

                Esta es la pantalla en la que se muestran las mesas disponibles del establecimiento. Se trata de un menú con 5 mesas:
                \begin{itemize}
                    \item 3 interiores
                    \item 2 terraza
                \end{itemize}
                Esta diferenciación se encuentra en la parte baja de la pantalla, la cual se destaca con dos colores diferentes.

                En pantalla se encuentran los botones que llevan a los activities de sus respectivas mesas y un botón para volver al activity anterior.
            \end{minipage}

        
    \newpage
    \section{APP Escritorio}
        % == Pantalla inicio de APP Escritorio ==
        \begin{minipage}{.5\textwidth}
            \vspace{0.5cm}
            \textbf{\begin{center}Inicio app\end{center}}
            \vspace{0.5cm}

            Esta es la pantalla inicial de la app escritorio, en la que el pulsar en \textbf{Entrar}, abre el siguiente activity.

            Cabe decir que es una app únicamente para el establecimiento de Taco Paco, ya que sus posibles usos son:
            \begin{itemize}
                \item Establecer un control de qué mesas están o no disponibles
                \item En caso de que no estén disponibles, se les puede cobrar su pedido desde esta app, lo que libera a conticuación la mesa
            \end{itemize}
        \end{minipage}
        \hspace{1cm}
        \begin{minipage}{.4\textwidth}
            \includegraphics[scale=0.1]{images/escritorio-principal.png}
        \end{minipage}

        % == Pantalla mesas ==
        \begin{minipage}{.5\textwidth}
            \vspace{0.5cm}
            \textbf{\begin{center}Control de mesas. CASO 1: Todas libres\end{center}}
            \vspace{0.5cm}

            Esta pantalla es la principal de la app. Como se puede observar contiene como contenido prioritario las 5 mesas des establecimiento. En este caso no hay ninguna mesa con un pedido pendiente.
    
        \end{minipage}
        \hspace{1cm}
        \begin{minipage}{.4\textwidth}
            \includegraphics[scale=0.1]{images/escritorio-mesas.png}
        \end{minipage}

        % == Pantalla mesas ocupadas ==
        \begin{minipage}{.5\textwidth}
            \vspace{0.5cm}
            \textbf{\begin{center}Control de mesas. CASO 2: Mesas ocupadas\end{center}}
            \vspace{0.5cm}

            Las mesas ocupadas se pueden apreciar por su fondo en rojo y el pedido que se debe en su interior.

            Para liberarla solo hay que clicar en el \textbf{botón azul} de cobrar
    
        \end{minipage}
        \hspace{1cm}
        \begin{minipage}{.4\textwidth}
            \includegraphics[scale=0.1]{images/escritorio-mesas.png}
        \end{minipage}

    \newpage
    \section{API Rest}

        \subsection{Flujo de la API con las APPs}
            \begin{figure}[h!]
                \centering
                \includegraphics[width=0.9\textwidth]{images/Diagrama-grafo.png}
                \caption{Diagrama de grafo}
            \end{figure}
        
        \newpage
        \subsection{Componentes}
            \begin{figure}[h!]
                \centering
                \includegraphics[width=0.9\textwidth]{images/estructura-api .png}
                \caption{Estructura de la API}
            \end{figure}

            \newpage
            \subsubsection{Models/}
            Explicación de lo que son los models
            
            % MODELO MESA
            \begin{tcolorbox} [colback=blue!5!white,colframe=gray!75!black,fonttitle=\bfseries,title=Modelo Mesa]
                \begin{lstlisting}
    const mongoose = require('mongoose')
    const mesaSchema = new mongoose.Schema(
        {
            nombre: { type: String, required: true, unique: true },
            ocupada: { type: Boolean, default: false }
        },
        { versionKey: false }
    );
    const Mesa = mongoose.model("Mesas", mesaSchema);
                \end{lstlisting}
            \end{tcolorbox}
 
            % MODELO PRODUCTO
            \begin{tcolorbox}[colback=blue!5!white,colframe=gray!75!black,fonttitle=\bfseries,title=Modelo Producto]
                \begin{lstlisting}
    const mongoose = require('mongoose')
    const pedidoSchema = new mongoose.Schema(
        {
            precioTotal: { type: Number, required: true }
        },
        { versionKey: false }
    );
    const Pedido = mongoose.model("Pedidos", pedidoSchema);
                \end{lstlisting}
            \end{tcolorbox}
            

    \newpage
    \section{Conclusión}


    % ================================= BIBLIOGRAFÍA =========================================
    \newpage
    \section{Bibliografía}
        \begin{itemize}
            \item \textbf{Formatos de imagen de Latex}: \url{https://manualdelatex.com/tutoriales/figuras}
            \item \textbf{Imagen al lado de texto en Latex}: \url{https://foro.rinconmatematico.com/index.php?topic=114810.0}
            \item \textbf{Texto al rededor de figuras en Lated}: \url{https://es.wikibooks.org/wiki/Manual_de_LaTeX/Inclusi%C3%B3n_de_gr%C3%A1ficos/Texto_alrededor_de_figuras}
            \item \textbf{Insertar código en el pdf}: \url{https://trspos.com/insertar-bloque-de-codigo-latex/}
        \end{itemize}
        




\end{document}